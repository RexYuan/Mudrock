\documentclass[12pt,a4paper]{llncs}

\usepackage{rex}

\usepackage{stmaryrd}
\usepackage[retainorgcmds]{IEEEtrantools}

% alternative caligraphy
\usepackage{calrsfs}
\DeclareMathAlphabet{\pazocal}{OMS}{zplm}{m}{n}

% alternative caligraphy
\DeclareFontFamily{U}{mathc}{}
\DeclareFontShape{U}{mathc}{m}{it}%
{<->s*[1.03] mathc10}{}
\DeclareMathAlphabet{\mathscr}{U}{mathc}{m}{it}

% minted python shortcut
\newminted{py}{frame=single,breaklines,autogobble,fontfamily=zi4,
escapeinside=||,mathescape=true,style=bw,linenos,python3=true}
\newmintedfile[pyfile]{py}{frame=single,breaklines,autogobble,fontfamily=zi4,
escapeinside=||,mathescape=true,linenos,python3=true}

\spnewtheorem{thing}{Thing}{\bfseries}{\itshape}

\newcommand{\cM}{\mathcal{M}}
\newcommand{\cT}{\mathcal{T}}
\newcommand{\MDNF}{\mathrm{M}_\mathrm{DNF}}
\newcommand{\MTERM}{\mathrm{M}_\mathrm{TERM}}

\begin{document}
\section*{Lambda' Algorithm}
For $f$
\begin{IEEEeqnarray*}{rCl}
f(x + a_k) & = & \bigvee_{f(a + a_k) = 1}^{m} a \\
\cM( f(x + a_k) ) & = &
\bigvee_{f(a + a_k) = 1}^{m} \cM(a) \\
& = & \bigvee_{f(a + a_k) = 1}^{m} \set{
    \bigwedge_{a[i] = 1} x_i
} \\
& = & \bigvee_{f(a + a_k) = 1}^{m} \cT'_k \\
\cM_{a_k} ( f ) & = &
(\bigvee_{f(a + a_k) = 1}^{m} \cT'_k)(x + a_k)
\end{IEEEeqnarray*}

For $S'_i = \set{v_n + a_i \mid n}$
\begin{IEEEeqnarray*}{rCl}
\MDNF(S'_i) & = & \set{
    \MTERM(v_n + a_i) \mid
    (v_n + a_i) \in S'_i
} \\
& = & \set{
    \bigwedge_{(v_n + a_i)[j] = 1} x_j \mid
    (v_n + a_i) \in S'_i
} \\
\bigvee_{(v_n + a_i) \in S'_i} \MDNF(S'_i) & = &
  \bigvee_{(v_n + a_i) \in S'_i}
    \bigwedge_{(v_n + a_i)[j] = 1} x_j \\
H_i & = &
(\bigvee_{(v_n + a_i) \in S_i} \MDNF(S'_i)) (x + a_i)
\end{IEEEeqnarray*}

\newpage
\section*{Proposition A'}
Let $f$ be a boolean function. If $a$ is an assignment
such that $f(a) = 1$,
then for the minterm DNF $\bigvee_{i=1}^m T_i$ of $f$
there is a term $T_a$ such that $\cM(T_a) = \MTERM(a)$.

\subsubsection*{Proof}
Assume
\begin{enumerate}
\item $v^\delta$ is a positive counterexample.
\item $I^\delta$ is non-empty.
\item $\MTERM(v^\delta + a_i) \in \cT'_i - \MDNF(S_i^{'\delta-1})$ for $i \in I^\delta$.
\item $\MDNF(S_i^\delta) \subseteq \cT_i$ for $i = 1 .. t$.
\item $H_i^\delta \to \cM_{a_i}(f)$ for $i = 1 .. t$.
\end{enumerate}

Then induction is as follows.
\begin{enumerate}[a.,start=3]

\item Let $i \in I^{\delta+1}$.
Since (b), $H_i^\delta (v^{\delta+1}) = 0$.
By definition,
\[
H_i^\delta (v^{\delta+1}) = 
(\bigvee_{v_n \in S_i^{'\delta}} \MDNF(S_i^{'\delta})) (v^{\delta+1} + a_i) = 0
\]
so
\[
\MTERM(v^{\delta+1} + a_i) \notin \MDNF(S_i^{'\delta})
\]
By (a),
\[
f(v^{\delta+1}) =
f((v^{\delta+1} + a_i) + a_i) = 1
\]
By Proposition A',
there is a term $T$ in the minterm DNF of
$f(x + a_i)$ such that
$\cM(T) = \MTERM(v^{\delta+1} + a_i)$.
Therefore
\[
\MTERM(v^{\delta+1} + a_i) \in \cT'_i
\]
This is the crux of the proof.
\addtocounter{enumi}{1}

\item By definition,
\[
H_i^{\delta+1} (x) =
(\bigvee_{(\widehat{v_n} + a_i) \in S_i^{\delta+1}}
\MDNF(S_i^{\delta+1}))
(x + a_i)
\]
and
\[
\cM_{a_k} ( f ) = (\bigvee_{n=1}^{s} \cT_k)(x + a_k)
\]
By (d), $\MDNF(S_i^{\delta+1}) \subseteq \cT_i$, so
\[
\bigvee_{(\widehat{v_n} + a_i) \in S_i^{\delta+1}}
\MDNF(S_i^{\delta+1}) \to
\bigvee_{n=1}^{s} \cT_k
\]
Therefore
\[
H_i^{\delta+1} (x) \to \cM_{a_k} ( f )
\]


\end{enumerate}












%\pyfile{../bshouty/cdnf.py}

\end{document}
